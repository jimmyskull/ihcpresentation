\documentclass[14pt,beamer]{beamer}
\usepackage[brazilian]{babel}
\usepackage[utf8]{inputenc} % Codificação UTF-8
\usepackage[T2A,T1]{fontenc} % Pacote de hifenização
\usepackage{graphicx} % Imagens

\usepackage{listings} % Mostrar códigos com cores
\usepackage{color}
\usepackage{textcomp}

\usefonttheme{professionalfonts}

%###############################################
%         PRECISA TER O RA NOS SLIDES?
%###############################################

\newcommand\cyrtext[1]{{\fontencoding{T2A}\selectfont #1}}

\usepackage{tabularx,colortbl} % Cores nas tabelas


\setbeamertemplate{title page}[default][colsep=-5bp,rounded=true]

\usetheme[height=15pt]{Berkeley}
\usecolortheme[RGB={26,89,142}]{structure} 
\setbeamertemplate{navigation symbols}{}
\setbeamertemplate{blocks}[rounded][shadow=true] 

\defbeamertemplate*{footline}{infolines theme without institution}
{
  \leavevmode%
  \hbox{%
  \begin{beamercolorbox}[wd=.5\paperwidth,ht=2.25ex,dp=1ex,center]{title in head/foot}%
    \usebeamerfont{title in head/foot}\insertshorttitle
  \end{beamercolorbox}%
  \begin{beamercolorbox}[wd=.5\paperwidth,ht=2.25ex,dp=1ex,right]{date in head/foot}%
    \usebeamerfont{date in head/foot}\insertshortdate{}\hspace*{7em}
    \insertframenumber{} / \inserttotalframenumber\hspace*{2ex}
  \end{beamercolorbox}}%
  \vskip0pt%
}



\title[Doutor Pecúlio]{Doutor Pecúlio - App para compras pelo menor preço}

\author[Carlos, Lucas e Paulo Roberto]{ \normalsize Carlos A. O. de Souza Junior \footnotesize{(570101439)} \\ 
		\and \normalsize Lucas J. C. Lorenzetti \footnotesize{(000000000)} \\ 
		\and \normalsize Paulo R. Urio \footnotesize{(570091403)} }
%\institute[]{
%  Departamento de Ciência da Computação \\
%  Universidade Estadual do Centro-Oeste \\
%  Guarapuava, Brasil
%}
\institute[]{
    \vspace{10px}~ \\
    \includegraphics[scale=0.2]{imagens/logo} \\
  {\rmfamily\scshape Universidade Estadual do Centro-Oeste}
}
\date{}

     


\makeatletter
\setlength{\beamer@headheight}{1cm}

\setbeamertemplate{footline}
{
  \leavevmode%
  \hbox{%
  \begin{beamercolorbox}[wd=.333333\paperwidth,ht=2.25ex,dp=1ex,center]{author in head/foot}%
    \usebeamerfont{author in head/foot}\insertshortauthor~~\beamer@ifempty{\insertshortinstitute}{}{\insertshortinstitute}
  \end{beamercolorbox}%
  \begin{beamercolorbox}[wd=.333333\paperwidth,ht=2.25ex,dp=1ex,center]{title in head/foot}%
    \usebeamerfont{title in head/foot}\insertshorttitle
  \end{beamercolorbox}%
  \begin{beamercolorbox}[wd=.333333\paperwidth,ht=2.25ex,dp=1ex,right]{date in head/foot}%
    \usebeamerfont{date in head/foot}\insertshortdate{\today}\hspace*{2em}
    \insertframenumber{} / \inserttotalframenumber\hspace*{2ex} 
  \end{beamercolorbox}}%
  \vskip0pt%
}

\makeatother

\begin{document}
	\begin{frame}
		\titlepage
	\end{frame}

\begin{frame}
	\frametitle{Conteúdo}
	\small
	\tableofcontents[hideallsubsections]
\end{frame}

% %%%%%%%%%%%%%%%%%%%%%%%%%%%%%%%% section %%%%%%%%%%%%%%%%%%%%%%%%%%%%%%%%%%%%
% -------------------------------- section ------------------------------------
\section{Introdução}

% --------------------------------- slide -------------------------------------
\begin{frame}
	\frametitle{Conteúdo}
	\small
	\tableofcontents[currentsection,hideothersubsections]
\end{frame}

% --------------------------------- slide -------------------------------------
\begin{frame}
	\frametitle{O aplicativo}
	
	\vspace{-30px}
	\begin{columns}
		\begin{column}{.3\textwidth}
			\begin{figure}
				\includegraphics[scale=.15]{imagens/docpig}
			\end{figure}
		\end{column}%
		\hfill%
		\begin{column}{9\textwidth}
			\large{\textbf{Doutor Pecúlio}}
		\end{column}%
	\end{columns}
	
	\vspace{10px}
	\begin{itemize}
		\item Me chamar de bebê é fácil, quero ver:
			\begin{itemize}
				\item colocar o peito pra fora; e
				\item deixa eu mamar. 	
			\end{itemize}
		\item Gata, você gosta de buceta? Não. Então me dá a tua?
		\item Acha que é bandida, mas nunca roubou um chocolate no mercado.
	\end{itemize}
\end{frame}

% %%%%%%%%%%%%%%%%%%%%%%%%%%%%%%%% section %%%%%%%%%%%%%%%%%%%%%%%%%%%%%%%%%%%%
% -------------------------------- section ------------------------------------
\section{Metas}

% --------------------------------- slide -------------------------------------
\begin{frame}
	\frametitle{Conteúdo}
	\footnotesize
	\tableofcontents[currentsection,hideothersubsections]
\end{frame}

\subsection{Usabilidade}
% --------------------------------- slide -------------------------------------
\begin{frame}
	\frametitle{Metas de usabilidade}

\begin{itemize}
	\item Mamãe, mamãe... me leva no circo? 
		\begin{itemize}
			\item Não, filho... Se querem te 
			ver, que venham aqui em casa...
		\end{itemize}
	\item O que é preciso para reunir os Beatles? 
	\begin{itemize}
		\item Mais duas balas.
	\end{itemize}
\end{itemize}

\end{frame}

\subsection{Experiência}
% --------------------------------- slide -------------------------------------
\begin{frame}
	\frametitle{Metas de experiência do usuário}

Qual a diferença entre um estudante português burro e um estudante português inteligente ?
\begin{description}
	\item[Português Burro] Copia tudo o que a professora escreve no
		quadro e quando ela apaga o quadro, ele apaga tudo no caderno.
	\item[Português Inteligente] Não copia nada porque ele sabe que a
	professora vai apagar mesmo.
\end{description}

\end{frame}

% %%%%%%%%%%%%%%%%%%%%%%%%%%%%%%%% section %%%%%%%%%%%%%%%%%%%%%%%%%%%%%%%%%%%%
% -------------------------------- section ------------------------------------
\section{Cenários}

% --------------------------------- slide -------------------------------------
\begin{frame}
	\frametitle{Conteúdo}
	\footnotesize
	\tableofcontents[currentsection,hideothersubsections]
\end{frame}

\subsection{Cenário 1}
% --------------------------------- slide -------------------------------------
\begin{frame}
	\frametitle{Cenário 1}

	O pai  vira-se para o filho mais velho:
	\begin{itemize}
		\item[--] Já fez?
		\item[--] Sim, papai!
		\item[--] E você, Jacó, já fez?
		\item[--] Sim, papai!
		\item[--] Sarah?
		\item[--] Já fiz, papai!
		\item[--] Raquel?
		\item[--] Também já fiz!
		\item[--] Ok! Então pode dar a descarga!
	\end{itemize}

\end{frame}

\subsection{Cenário 2}
% --------------------------------- slide -------------------------------------
\begin{frame}
	\frametitle{Cenário 2}

Um cara vai se confessar com um padre e diz:
	\begin{itemize}
		\item[--] Padre, fiz amor com uma preta num quarto escuro... É pecado?
		\item[--]  Não, meu filho. É pontaria!
	\end{itemize}
\end{frame}

\subsection{Cenário 3}
% --------------------------------- slide -------------------------------------
\begin{frame}
	\frametitle{Cenário 3}
O orifício circular corrugado, localizado na parte ínfero-lombar da região glútea de um indivíduo em alto grau etílico, deixa de estar em consonância com os ditames referentes ao direito individual de propriedade.

\end{frame}

% %%%%%%%%%%%%%%%%%%%%%%%%%%%%%%%% section %%%%%%%%%%%%%%%%%%%%%%%%%%%%%%%%%%%%
% -------------------------------- section ------------------------------------
\section{Casos de uso}

% --------------------------------- slide -------------------------------------
\begin{frame}
	\frametitle{Conteúdo}
	\footnotesize
	\tableofcontents[currentsection,hideothersubsections]
\end{frame}

\subsection{Caso de uso 1}
% --------------------------------- slide -------------------------------------
\begin{frame}
	\frametitle{Caso de uso 1}
\end{frame}

\subsection{Caso de uso 2}
% --------------------------------- slide -------------------------------------
\begin{frame}
	\frametitle{Caso de uso 2}
\end{frame}

% %%%%%%%%%%%%%%%%%%%%%%%%%%%%%%%% section %%%%%%%%%%%%%%%%%%%%%%%%%%%%%%%%%%%%
% -------------------------------- section ------------------------------------
\section{Requisitos}

% --------------------------------- slide -------------------------------------
\begin{frame}
	\frametitle{Conteúdo}
	\small
	\tableofcontents[currentsection,hideothersubsections]
\end{frame}

% --------------------------------- slide -------------------------------------
\begin{frame}
	\frametitle{Requisitos do sistema}
\end{frame}

% %%%%%%%%%%%%%%%%%%%%%%%%%%%%%%%% section %%%%%%%%%%%%%%%%%%%%%%%%%%%%%%%%%%%%
% -------------------------------- section ------------------------------------
\section{Modelo e Protótipo}

% --------------------------------- slide -------------------------------------
\begin{frame}
	\frametitle{Conteúdo}
	\footnotesize
	\tableofcontents[currentsection,hideothersubsections]
\end{frame}

% --------------------------------- slide -------------------------------------
\begin{frame}
	\frametitle{Diagrama UML}

\end{frame}

\subsection{Baixo nível}
% --------------------------------- slide -------------------------------------
\begin{frame}
	\frametitle{Protótipo de baixo nível}

\end{frame}

\subsection{Alto nível}
% --------------------------------- slide -------------------------------------
\begin{frame}
	\frametitle{Protótipo de alto nível}

\end{frame}

% %%%%%%%%%%%%%%%%%%%%%%%%%%%%%%%% section %%%%%%%%%%%%%%%%%%%%%%%%%%%%%%%%%%%%
% -------------------------------- section ------------------------------------
\section{Conclusão}

% --------------------------------- slide -------------------------------------
\begin{frame}
	\frametitle{Conteúdo}
	\small
	\tableofcontents[currentsection,hideothersubsections]
\end{frame}

% --------------------------------- slide -------------------------------------
\begin{frame}
	\frametitle{Considerações Finais}

	\begin{block}{}
	Estou cagando e andando.
	\end{block}
\end{frame}

% %%%%%%%%%%%%%%%%%%%%%%%%%%%%%%%% section %%%%%%%%%%%%%%%%%%%%%%%%%%%%%%%%%%%%
% ------------------------------ slide ----------------------------------

\section*{Referências}

\begin{frame}{Referências}

	\footnotesize{
	\begin{thebibliography}{99}
	\beamertemplatebookbibitems
	\bibitem[Barbosa]{p1} BARBOSA, Simone D. J.; SILVA, Bruno S. (2010)
	\newblock Interação humano-computador
	\newblock \emph{Editora Campus-Elsevier}, ISBN 978-85-352-3418-3.

	\bibitem[Rocha]{p1} DA ROCHA, H. V.; BARANAUSKAS, M. C. C. (2010)
	\newblock Design e avalia{\c{c}}{\~a}o de interfaces humano-computador
	\newblock \emph{Unicamp}, ISBN 978-85-888-3304-3.
	
	\bibitem[Love]{p1} LOVE, Steve (2005)
	\newblock Understanding mobile human-computer interaction
	\newblock \emph{Elsevier Science}, ISBN 978-00-804-5580-8.
	\end{thebibliography}
	}
%	\beamertemplatearticlebibitems

\end{frame}

\section*{}

\begin{frame}
	\vspace{-15px}
	\begin{center}
		\Huge{Obrigado!}

		\small

		\cyrtext{Спасибо за внимание!}
	\end{center}

	\vspace{15px}
	\small Carlos Alberto Oliveira de Souza Junior \\
	\footnotesize{  \textsl{carlos\_bertojr@hotmail.com} }
	
	\vspace{20px}
	\small Lucas José Campos Lorenzetti \\
	\footnotesize{  \textsl{goodlucas@gmail.com} }
	
	\vspace{20px}
	\small Paulo Roberto Urio \\
	\footnotesize{  \textsl{paulo@bk.ru} }
\end{frame}

\end{document}
